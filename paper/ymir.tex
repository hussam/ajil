\subsection{Ymir's thoughts}

\begin{itemize}
\item
 I don't know yet what the best ordering is for this material. Should policy be introduced in a general fashion before
  presenting the protocol, or should the protocol be displayed and the policy section explains its flexibility?
\item
 Put examples of how the reaction domain could be specified, while representing some policy that 'makes sense'.
  For instance, "Tax everybody equally", and "Tax richest nodes in the group that has most traffic".
\item
 The time duration for the drop/delay policy is unspecified yet. There are two or three different approaches that make sense.
  One is to keep it as it is and be totally agnostic. 
  Second is to gradually decrease the limit using an exponential average
  as with the gamma parameter. This will behave a lot like TCP, always trying to be close to L. The key point here is that
  we don't really know the demand, we just predict it from previous usage of multicast. Thus if a lot of people leave, of
  course the others should be allowed to increase their multicast traffic rates.
  Third option is to change the monitor code slightly so we report rate and 'desired rate' of each group, meaning how fast
  the group would want to be if it didn't have to slow down.
  Then we can evaluate the policy for reaction when the /desired limit/ reaches the 95\% of L mark, otherwise allow everybody
  to send as fast as they want. Desired rate would be calculated by interpolating the current amount of bytes or packets 
  a node wanted to send (whether or not they were dropped) versus the time spent sending, and not spent waiting (in case 
  they were delayed). 
  Fourth option is to have the time decay thing so that we tenure delays that have been happening for a long time. I don't
  really know what would be a good recovery or an implementation of this approach, so I think we should try something else.
  Three sounds best to me, and the second option might also work.
\item
  We need to capture some notion of how fast rates are changing, like Hussam and I have been talking about.
  Ideally, we'd like to be able to say something like "assuming rates cannot more than double during one epoch", and
  deduce that we can respond fast to that. This also means that epoch sizes need to be non-trivial, otherwise they're very
  choppy. It may be worth it to look at the burstiness of various protocols, maybe if there's something on SRM, to see what
  rates are reasonable and recommend a good value of E. 

\item 
  Try to make sure that the 'criticial threshold' thing is clear. We have a limit $L$ that should never be exceeded. 
  We use $L/c$ of the traffic on average for broadcast channel traffic. That gives $(1-\frac{1}{c})L$ of remaining 
  bandwidth. We define a critical threshold to be something like $0.95 L$, or the point at which there is a crisis,
  so it needs to be clear what $L$ we are referring to here. Should this 95\% be configurable? If it is, you need to
  bind it to the rate at which rates change (yeah, that's the rate of a rate) -- if things can change fast, you need
  a bigger cushion. Perhaps you can move this responsibility on to the administrator and just tell him what it does.

\end{itemize}